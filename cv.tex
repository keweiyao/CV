%%%%%%%%%%%%%%%%%%%%%%%%%%%%%%%%%%%%%%%%%
% Medium Length Professional CV
% LaTeX Template
% Version 2.0 (8/5/13)
%
% This template has been downloaded from:
% http://www.LaTeXTemplates.com
%
% Original author:
% Trey Hunner (http://www.treyhunner.com/)
%
% Important note:
% This template requires the resume.cls file to be in the same directory as the
% .tex file. The resume.cls file provides the resume style used for structuring the
% document.
%
%%%%%%%%%%%%%%%%%%%%%%%%%%%%%%%%%%%%%%%%%

%----------------------------------------------------------------------------------------
%	PACKAGES AND OTHER DOCUMENT CONFIGURATIONS
%----------------------------------------------------------------------------------------

\documentclass{resume} % Use the custom resume.cls style

\usepackage[left=0.75in,top=0.6in,right=0.75in,bottom=0.6in]{geometry} % Document margins
\usepackage{hyperref}

\name{Weiyao Ke} % Your name
\address{wk42@phy.duke.edu \\ \href{https://webhome.phy.duke.edu/~wk42}{https://webhome.phy.duke.edu/$\sim$wk42}} % Your phone number and email

\begin{document}

\begin{rSection}{Research Interest}
\vspace{.3em}
\begin{Article}
\item Transport theory of the hot and dense QCD matter.
\vspace{.3em}
\item Monte Carlo simulations of partonic transport and the treatment of coherence effects.
\vspace{.3em}
\item Computer modeling of the dynamics of ultra-relativistic heavy-ion collisions.
\vspace{.3em}
\item Application of machine learning and Bayesian techniques to nuclear physics.
\vspace{.3em}
\item Phenomenology models of QCD phase transition.
\end{Article}
%------------------------------------------------
\end{rSection}


\begin{rSection}{Education}
{\bf Duke University} \hfill {\em August 2014 - present} \\ 
Physics, PhD candidate \\
Supervisor: Prof. Steffen A. Bass

{\bf Peking University} \hfill {\em September 2010 - June 2014} \\ 
B.S. in Physics \\
Supervisor: Prof. Yuxin Liu \\
Thesis: {\it QCD Phase Transitions in Phenomenology Models. }
\end{rSection}

\begin{rSection}{Publications}
\vspace{.3em}
\begin{Article}
\item Weiyao Ke, Yingru Xu, and Steffen A. Bass {\it Modeling of quantum-coherence effects in parton radiative energy loss}, arXiv:1810.08177.
\vspace{.3em}
\item Weiyao Ke, Yingru Xu, and Steffen A. Bass {\it A linearized Boltzmann-Langevin transport model for heavy quark transport in hot and dense QCD matter}, arXiv:1806.08848 (submitted to Physical Review C).
\vspace{.3em}
\item Weiyao Ke, J. Scott Moreland, Jonah E. Bernhard, and Steffen A. Bass {\it Constraints on rapidity-dependent initial conditions from charged-particle pseudorapidity densities and two-particle correlations}, Phys. Rev. C {\bf 96}, 044912 (2017).
\vspace{.3em}
\item Weiyao Ke and Yu-xin Liu, {\it Interfact Tension and Interface Entropy in the 2+1 Flavor Nambu--Jona-Lasinio Model}, Phys. Rev. D {\bf 89}, 074041 (2014).
\end{Article}
\end{rSection}


\begin{rSection}{Collaboration Publications}
\vspace{.3em}
\begin{Article}
\item Shanshan Cao {\it et al}, {\it Towards the extraction of heavy-quark transport coefficients in quark-gluon plasma}, arXiv:1809.07894.
\end{Article}
\end{rSection}


\begin{rSection}{Conference Presentations and Proceedings}
\begin{Conference}{Hard Probes 2018}{Aix-les-Bains, Fracne}
\item \underline{Weiyao Ke}*, Yingru Xu, and Steffen Bass, {\it $\hat{q}$ analysis in a hybrid Boltzmann-Langevin approach with an improved LPM treatment}. 
\end{Conference}

\begin{Conference}{Quark Matter 2018}{Lido, Italy}
\item \underline{Weiyao Ke}*, Yingru Xu, and Steffen Bass, {\it A hybrid transport model of heavy quark evolution in a QGP medium}, poster presentation.
\item Xiaojun Yao*, \underline{Weiyao Ke}, Yingru Xu, Steffen Bass, and Berndt M\"uller, {\it Quarkonium production in heavy-ion collisions: coupled Boltzmann transport equations}, arXiv:1807.06199. 
\end{Conference}

\begin{Conference}{APS Fall Meeting of the division of Nuclear Physics 2017}{Pittsburgh, USA}
\item \underline{Weiyao Ke}*, J. Scott Moreland, Jonah Bernhard, and Steffen Bass, {\it Constraints on rapidity-dependent initial conditions from charged particle pseudorapidity densities and correlations at the LHC}, Nuclear and Particle Physics Proceedings 289, 483-486.
\end{Conference}

\begin{Conference}{Quark Matter 2017}{Chicago, USA}
\item {Weiyao Ke}*, J. Scott Moreland, Jonah Bernhard, and Steffen Bass, {\it Constraints on 3D hydro initial conditions from experimental data and systematic predictions of longitudinal observables}, poster presentation.
\item J. Scott Moreland*, Jonah Bernhard, \underline{Weiyao Ke}, and Steffen Bass, {\it Flow in small and large quark-gluon plasma droplets: the role of nucleon substructure}, Nuclear Physics A 967, 361-364.
\end{Conference}

\begin{Conference}{Hard Probes 2016}{Wuhan, China}
\item \underline{Weiyao Ke}*, J. Scott Moreland, Jonah Bernhard, and Steffen Bass, {\it Constraints on rapidity-dependent initial conditions from charged particle pseudorapidity densities and two-particle correlations at LHC}, poster presentation.
\end{Conference}

\begin{Conference}{Quark Matter 2015}{Kobe, Japan}
\item \underline{Weiyao Ke}*, J. Scott Moreland, Shanshan Cao, and Steffen Bass, {\it Sensitivity of Heavy Quark Observables to Event Geometries in Pb + Pb Collisions}, poster presentation.
\end{Conference}

\begin{Conference}{Hard Probes 2015}{Montr\'eal, Canada}
\item \underline{Weiyao Ke}*, J. Scott Moreland, and Steffen Bass, {\it Heavy Quark Initial Condition Consistent with Soft Matter Production}, poster presentation.
\item Yingru Xu*, Shanshan Cao, Marlene Nahrgang, \underline{Weiyao Ke}, Guang-You Qin, Jussi Auvinen, and Steffen Bass, {\it Heavy-flavor dynamics in relativistic p-Pb collisions at $\sqrt{s_{NN}}=5.02$ TeV}, Nuclear and Particle Physics Proceedings 276, 225-228.
\end{Conference}

\end{rSection}


%----------------------------------------------------------------------------------------
%	EXAMPLE SECTION
%----------------------------------------------------------------------------------------

%\begin{rSection}{Section Name}

%Section content\ldots

%\end{rSection}

%----------------------------------------------------------------------------------------

\end{document}
