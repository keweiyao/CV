%%%%%%%%%%%%%%%%%%%%%%%%%%%%%%%%%%%%%%%%%
% Medium Length Professional CV
% LaTeX Template
% Version 2.0 (8/5/13)
%
% This template has been downloaded from:
% http://www.LaTeXTemplates.com
%
% Original author:
% Trey Hunner (http://www.treyhunner.com/)
%
% Important note:
% This template requires the resume.cls file to be in the same directory as the
% .tex file. The resume.cls file provides the resume style used for structuring the
% document.
%
%%%%%%%%%%%%%%%%%%%%%%%%%%%%%%%%%%%%%%%%%

%----------------------------------------------------------------------------------------
%	PACKAGES AND OTHER DOCUMENT CONFIGURATIONS
%----------------------------------------------------------------------------------------

\documentclass{resume} % Use the custom resume.cls style

\usepackage[left=0.75in,top=0.6in,right=0.75in,bottom=0.6in]{geometry} % Document margins

\name{Weiyao Ke} % Your name
\address{\it Department of Physics, Duke University, Durham, NC 27708, USA} % Your address
\address{ wk42@phy.duke.edu} % Your phone number and email

\begin{document}

%----------------------------------------------------------------------------------------
%	EDUCATION SECTION
%----------------------------------------------------------------------------------------

\begin{rSection}{Education}

{\bf Duke University} \hfill {\em August 2014 - present} \\ 
Physics, PhD candidate \hfill Prof. Steffen A. Bass

{\bf Peking University} \hfill {\em September 2010 - June 2014} \\ 
B.S. in Physics \hfill Prof. Yuxin Liu

\end{rSection}

%----------------------------------------------------------------------------------------
%	WORK EXPERIENCE SECTION
%----------------------------------------------------------------------------------------

\begin{rSection}{Research}

\begin{rSubsection}{Duke University}{2014 - Present}{}{}
\item The development of a Langevin plus linear Boltzmann equation transport model for heavy quark propagation inside a quark-gluon plasma.
\item Parametric three-dimensional event-by-event initial condition model for relativistic heavy-ion collisions.
\item Model-to-data comparison using Bayesian methodology and model parameters inference.

%------------------------------------------------

\end{rSubsection}
\begin{rSubsection}{Peking University}{2013 - 2014}{}{}
\item Study of QCD phase transitions in Nambu-Jona-Lasinio model and Polyakov loop improved Nambu-Jona-Lasinio model.
\end{rSubsection}

%------------------------------------------------
\end{rSection}


\begin{rSection}{Conference}

\begin{rSubsection}{APS Fall Meeting of the division of Nuclear Physics 2017}{Pittsburgh, USA}{}{}
\item {Presentation: \it Constrain initial 3D entropy production in relativistic heavy-ion collisions at the LHC}.
\end{rSubsection}

\begin{rSubsection}{Quark Matter 2017}{Chicago, USA}{}{}
\item {Poster: \it Constraints on 3D hydro initial conditions from experimental data and systematic predictions of longitudinal observables}.
\end{rSubsection}

\begin{rSubsection}{Hard Probes 2016}{Wuhan, China}{}{}
\item {Poster and flash talk: \it Constraints on rapidity-dependent initial conditions from charged particle pseudorapidity densities and two-particle correlations at LHC}.
\end{rSubsection}

\begin{rSubsection}{Quark Matter 2015}{Kobe, Japan}{}{}
\item {Poster: \it Sensitivity of Heavy Quark Observables to Event Geometries in Pb + Pb Collisions}.
\end{rSubsection}

\begin{rSubsection}{Hard Probes 2015}{Montr\'eal, Canada}{}{}
\item {Poster: \it Heavy Quark Initial Condition Consistent with Soft Matter Production}.
\end{rSubsection}

\end{rSection}

\begin{rSection}{Publication}
Weiyao Ke, J. Scott Moreland, Jonah E. Bernhard, and Steffen A. Bass {\it Constraints on rapidity-dependent initial conditions from charged-particle pseudorapidity densities and two-particle correlations}, Phys. Rev. C {\bf 96}, 044912 (2017).

Weiyao Ke and Yu-xin Liu, {\it Interface Tension and Interface Entropy in the 2+1 Flavor Nambu--Jona-Lasinio Model}, Phys. Rev. D {\bf 89}, 074041 (2014).

\end{rSection}

%----------------------------------------------------------------------------------------
%	EXAMPLE SECTION
%----------------------------------------------------------------------------------------

%\begin{rSection}{Section Name}

%Section content\ldots

%\end{rSection}

%----------------------------------------------------------------------------------------

\end{document}
